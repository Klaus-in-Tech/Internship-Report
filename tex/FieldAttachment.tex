\chapter{Field Attachment Activities}
During my internship training at BPW, I did a number of activities such as Computer and Phone hardware maintenance, Software installations of operating systems and Drivers of all devices which were missing on computers and making updates, anti­viruses (Kaspersky installations), Installation and replacement of faulty components, Troubleshooting networks and computer hardware components, Digital Marketing\cite{Digital} etc. \\
During this period, I was able to gain more experience in this field of IT and add on my skills, improve on my socializing and interpersonal skills. 
\section{Chatbot Development}
 A computer program designed to simulate conversation with human users, especially over the Internet. \\ \\
\noindent It is an assistant that communicates with us through text messages, a virtual companion that integrates into websites, applications or instant messengers and helps entrepreneurs to get closer to customers. Such a bot is an automated system of communication with users. \\ \\
\noindent During internship I developed a simple messenger chatbot. \\ \\
\subsection{Types of Chatbots}
\normalsize{\bf Simple chatbots} \\
They work based on pre-written keywords that they understand. Each of these commands must be written by the developer separately using regular expressions or other forms of string analysis. If the user has asked a question without using a single keyword, the robot can not understand it and, as a rule, responds with messages like “sorry, I did not understand” .\\
\normalsize{\bf Smart chatbots} \\
They rely on artificial intelligence when they communicate with users. Instead of pre-prepared answers, the robot responds with adequate suggestions on the topic. In addition, all the words said by the customers are recorded for later processing.\\
\subsection{Importances of Chatbots to a company.}
\normalsize{\bf Productivity} \\
Chatbots provide the assistance or access to information quickly and efficiently.\\
\normalsize{\bf Entertainment} \\
Chatbots amuse people by giving them funny tips, they also help killing time when users have nothing to do.\\
\normalsize{\bf Social and relational factors} \\
Chatbots fuel conversions and enhance social experiences. Chatting with bots also helps to avoid lonliness, gives a chance to talk without being judged and improves conversational skills.\\
\normalsize{\bf Curiosity} \\
The novelty of chatbots sparks curiosity. People want to explore their abilities and to try something new.\\
\subsection{Steps I took in developing the chatbot.}
\begin{itemize}
\item I first understood the potential users of the bot through the FAQs about the company products.
\item Defined my goals which included increasing client engagement.
\item Picked a platform which I used in the modelling of the chatbot.
\item Designed a conversation flow between the bot and the user\cite{chatbot}.
\begin{figure}[h!]
\begin{center}
	\includegraphics[scale= 0.4]{img/chatfuel.png}
	\caption{Showing the conversation flow }
	\label{fig:symbols}
\end{center}
\end{figure}

\end{itemize}

\section{Phone Repair}

\subsection{Mobile Phone Repairing Tools and Equipment}
During my internship at BPW my supervisor introduced me to their phone technician who really exposed me to alot of experiences concerning both phone hardware and software.
There many tools used in mobile phone repairing but below some of the tools that I used\cite{phone}


\begin{itemize}
\item[1.] \textbf{Soldering Iron:} Used for soldering.
\begin{figure}[h!]
\begin{raggedright}
	\includegraphics[scale=0.5]{img/solder.PNG}
	\caption{Soldering Iron}
	\label{fig:symbols}
\end{raggedright}
\end{figure}

\item[2.] \textbf{Multimeter:} To check PCB track and electronic Components.
\begin{figure}[h!]
\begin{raggedright}
	\includegraphics[scale=0.5]{img/multimeter.PNG}
	\caption{Multimeter}
	\label{fig:symbols}
\end{raggedright}
\end{figure}
\begin{figure}[h!]
\begin{center}
	\includegraphics[scale= 0.1]{img/meter.jpg}
	\caption{Me using the multimeter }
	\label{fig:symbols}
\end{center}
\end{figure}


\newpage
\item[3.] \textbf{Screwdriver:} Used to Remove and Tighten Screws from Mobile Phone.
\begin{figure}[h!]
\begin{raggedright}
	\includegraphics[scale=0.5]{img/screw.PNG}
	\caption{Screw Driver}
	\label{fig:symbols}
\end{raggedright}
\end{figure}

\item[4.] \textbf{Paste Flux:} Used While Soldering.
\begin{figure}[h!]
\begin{raggedright}
	\includegraphics[scale=0.5]{img/paste.jpg}
	\caption{Screw Driver}
	\label{fig:symbols}
\end{raggedright}
\end{figure}

\end{itemize}

\subsection{Identification of PCB}
The most important part of the phone is the PCB(Printed Circuit Board) which is like the motherboard in the PC so we had to first name the different parts on the PCB and the common faults they normally have.\\
\begin{figure}[h!]
\begin{center}
	\includegraphics[scale=0.9]{img/PCB.PNG}
	\caption{Printed Circuit Board.}
	\label{fig:symbols}
\end{center}
\end{figure}
\subsubsection{Definition of different Parts of the PCB}
\begin{itemize}
\item[1.] \textbf{Antenna Switch:} It is found in the Network Section of a Mobile Phone and is made up of metal and non-metal. In GSM sets it is found in white colour and in CDMA sets it is found in golden metal.\\ 
\textbf{Function:} It searches network and passes forward after tuning.\\
\textbf{Fault:} If the Anteena Switch is faulty then there will be no network in the mobile phone.

\item[2.] \textbf{P.F.O:} It is found near the Anteena Switch in the Network Section of a Mobile Phone. It is also called P.A (Power Amplifier) and Band Pass Filter.\\
\textbf{Function:} It filters and amplifies network frequency and selects the home network.\\
\textbf{Fault:} If the PFO is faulty then there will be no network in the mobile phone. If it gets short then the mobile phone will get dead.

\item[3.] \textbf{RF IC / Hager / Network IC:} It is found near the PFO in the Network Section of a Mobile Phone. It is also called RF signal processor.\\
\textbf{Function:} It works as transmitter and receiver of audio and radio waves according to the instruction from the CPU.\\
\textbf{Fault:} If the RF IC is faulty then there will be problem with network in the mobile phone. Sometimes mobile phone can even get dead.

\item[4.] \textbf{26 MHz Crystal Oscillator:} It is found near the PFO in the Network Section of a Mobile Phone. It is also called Network Crystal. It is made up of metal.\\
\textbf{Function:} It creates frequency during outgoing calls.\\
\textbf{Fault:} If this crystal is faulty then there will be no outgoing call and no network in the mobile phone.

\item[5.] \textbf{VCO:} It is found near the Network IC in the Network Section of a Mobile Phone.\\
\textbf{Function:} It sends time, date and voltage to the RF IC / Hager and the CPU. It also creates frequency after taking command from the CPU.\\
\textbf{Fault:} If it is faulty then there will be no network in the mobile phone and it will display “Call End” or “Call Failed”.

\item[6.] \textbf{RX Filter:} It is found in the Network Section of a Mobile Phone.\\
\textbf{Function:} It filters frequency during incoming calls.\\
\textbf{Fault:} If it is faulty then there will network problem during incoming
calls.

\item[7.] \textbf{TX Filter:} It is found in the Network Section of a Mobile Phone.\\
\textbf{Function:} It filters frequency during outgoing calls.\\
\textbf{Fault:} If it is faulty then there will network problem during outgoing calls.

\item[8.] \textbf{ROM:} It is found in the Power Section of a Mobile Phone.\\
\textbf{Function:} It loads current operating program in a Mobile Phone.\\
\textbf{Fault:} If ROM is faulty then there will software problem in the mobile phone and the set will get dead.

\item[9.] \textbf{RAM:} It is found in the Power Section of a Mobile Phone.\\
\textbf{Function:} It sends and receives commands of the operating program in a mobile phone.\\
\textbf{Fault:} If RAM is faulty then there will be software problem in the mobile phone and it will get frequently get hanged and the set can even get dead.

\item[10.] \textbf{Flash IC:} It is found in the Power Section of a Mobile Phone. It is also called EEPROM IC, Memory IC, RAM IC and ROM IC.\\
\textbf{Function:} Software of the mobile phone is installed in the Flash IC.
\textbf{Fault:} If Flash IC is faulty then the mobile phone will not work properly and it can even get dead.

\item[11.] \textbf{Power IC:} It is found in the Power Section of a Mobile Phone. There are many small components mainly capacitor around this IC. RTC is near the Power IC.\\
\textbf{Function:} It takes power from the battery and supplies to all other parts of a mobile phone.\\
\textbf{Fault:} If Power IC is faulty then the set will get dead.

\item[12.] \textbf{Charging IC:} It is found in the Power Section near R22.\\
\textbf{Function:} It takes current from the charger and charge the battery.\\
\textbf{Fault:} If Charging IC is faulty then the set will not get charged. If theCharging IC is short then the set will get dead.

\item[13.] \textbf{RTC (Simple Silicon Crystal):} It is found in the Power Section near Power IC. It is made up of either metal or non-metal. It is of long shape.\\
\textbf{Function:} It helps to run the date and time in a mobile phone.\\
\textbf{Fault:} If RTC is faulty then there will be no date or time in the mobile phone and the set can even get dead.

\item[14.] \textbf{CPU:} It is found in the Power Section. It is also called MAD IC, RAP IC and UPP. It is the largest IC on the PCB of a Mobile Phone and it looks different from all other ICs.\\
\textbf{Function:} It controls all sections of a mobile phone.\\
\textbf{Fault:} If CPU is faulty then the mobile phone will get dead.

\item[15.] \textbf{Logic IC / UI IC:} It is found in any section of a mobile phone. It has 20 pins or legs. It is also called UI IC and Interface IC.\\
\textbf{Function:} It controls Ringer, Vibrator and LED of a mobile phone.\\
\textbf{Fault:} If Logic IC / UI IC is faulty then Ringer, Vibrator and LED of mobile phone will nor work properly.

\item[16.] \textbf{Audio IC:} It is found in Power Section of a mobile phone. It is alsocalled Cobba IC and Melody IC.\\
\textbf{Function:} It controls Speaker and Microphone of a mobile phone.
\textbf{Fault:} If Audio IC is faulty then Speaker and Microphone of a mobile phone will not work and the set can even get dead.

\end{itemize}

\subsection{Common phone Faults and solutions}
\begin{itemize}
\item[1.] \textbf{Ringer:} Type of component that rings or plays loud sound is called Ringer. It is also called by several other names like – I.H.F Speaker, Buzzer, Melody etc.
\begin{table}[!ht]
\centering
\begin{tabular}{|p{2.5in}|p{3.5in}|}
\hline
\textbf{Faults} & \textbf{Possible Soultions} \\ \hline
Ringer not working & Check Ringer Settings in Mobile Phone. Check Ringer Volume and Silent Mode.\\ 
Less sound from the Ringer & Open Mobile Phone and Clean Ringer Point and Ringer Connector. \\ Sound coming from Ringer but with interruption & Check Ringer by Keeping the Multimeter in Buzzer Mode. Value must be 8-10 Ohm. If theValue is not between 8-10 Ohm then change the Ringer. \\ 
Sound not clear & Check Track of Ringer Section. Do Jumper Wherever required \\
\hline
\end{tabular}
\caption{Faults \& Solutions of the Ringer.}
\end{table}

\item[2.] \textbf{Vibrator:} Type of component that vibrates. It is also called Motor.Vibrator is controlled by Logic IC or Power IC.
\begin{table}[!ht]
\centering
\begin{tabular}{|p{2.5in}|p{3.5in}|}
\hline
\textbf{Faults} & \textbf{Possible Soultions} \\ \hline
Vibrator not working & Check Vibrator Settings in Mobile Phone. Check if Vibrator is ON or OFF.\\ 
Vibration with interruption & Open Mobile Phone and Clean Vibrator Tips Connector. \\ 
Vibration Hangs & Check Vibrator by Keeping the Multimeter in Buzzer Mode. Value must be 8-16 Ohm. If the Value is not between 8-16 Ohm then change the Vibrator / Motor.\\ 
                & Check Track of Vibrator Section. Do Jumper Wherever required.\\
\hline
\end{tabular}
\caption{Faults \& Solutions of the Vibrator.}
\end{table}

\item[3.] \textbf{Light:} Type of component that generates light in the Mobile Phone. These are generally LED or Light Emitting Diode.
\begin{table}[!ht]
\centering
\begin{tabular}{|p{2.5in}|p{3.5in}|}
\hline
\textbf{Faults} & \textbf{Possible Soultions} \\ \hline
No Light & Check Light Settings.\\ 
Light in only Keypad or Display & Keep Multimeter in Buzzer Mode and Check LED. If LED is Good then it will Glow.If LED is Faulty then it will Not Glow. \\ 
Some lights not working & Change LED or Jumper.\\ 
                & Check Boosting Coil and Change if Required.\\
                & Change Display and Check.\\
\hline
\end{tabular}
\caption{Faults \& Solutions of the LED.}
\end{table}

\newpage
\item[4.] \textbf{Earpiece:} Type of component that helps to listen to sound during phone call. It is also called Speaker or Ear Speaker. Earpiece is controlled by Audio IC or Power IC (UEM).
\begin{table}[!ht]
\centering
\begin{tabular}{|p{2.5in}|p{3.5in}|}
\hline
\textbf{Faults} & \textbf{Possible Soultions} \\ \hline
No sound during phone call & Check Speaker Volume during Phone Call.\\ 
Less sound during phone call & Check Earpiece / Speaker by Keeping the Multimeter in Buzzer Mode. Value must be 25-35 Ohm. If the Value is not between 25-35 Ohm then change the Earpiece / Speaker.
Check Track of Earpiece Section. Do Jumper Wherever required. \\ 
Sound with interruption & Check Track of Earpiece Section. Do Jumper Wherever required.\\ 
                & UEM / Audio IC: Heat, Reball or Change.\\
                & CPU: Heat, Reball or Change.\\
\hline
\end{tabular}
\caption{Faults \& Solutions of the Earpiece.}
\end{table}

\item[5.] \textbf{Screen Touch:} Type of component that helps to operate a mobile phone by touching the screen. Touch Screen is available in different sizes. It normally has 4 Points Namely: - (+), (-), (RX), (TX). Screen Touch is also called PDA. It is controlled by the CPU. In some Mobile Phones there is an Interface IC called PDA IC or Screen Touch IC.
\begin{table}[!ht]
\centering
\begin{tabular}{|p{2.5in}|p{3.5in}|}
\hline
\textbf{Faults} & \textbf{Possible Soultions} \\ \hline
Screen Touch not Working & Check Settings if the Mobile Phone has Both Keypad and Touch Screen.\\ 
Only Half Screen Touch Works & Clean and Resold PDA Tips and PDA Connector. \\ 
One key is pressed and some other key works & Change PDA.\\ 
                & Check Track of the PDA Section and Jumper if Required.\\
                & CPU: Heat, Reball or Change.\\
\hline
\end{tabular}
\caption{Faults \& Solutions of the Screen Touch.}
\end{table}
\newpage
\item[6.] \textbf{Microphone:} Type of component that helps to transmit sound from one mobile phone
to another during phone call.
\begin{table}[!ht]
\centering
\begin{tabular}{|p{2.5in}|p{3.5in}|}
\hline
\textbf{Faults} & \textbf{Possible Soultions} \\ \hline
No sound or Less Sound during phone call & Check Microphone settings.\\ 
Sound with interruption or Changed sound & Check and clean Microphone Tips and Connector. \\ 
                & Check Microphone by Keeping the Multimeter in Buzzer Mode. Value must be 600-1800
Ohm. If the Value is not between 600-1800 Ohm then change the Microphone.\\ 
        
\hline
\end{tabular}
\caption{Faults \& Solutions of the Microphone.}
\end{table}

\item[7.] \textbf{Headphone:} Type of component that does the job of Mic and Speaker separately.
When we insert Headphone, then Speaker and Microphone of the Mobile Phone Gets Disconnected. Headphone is controlled by C.P.U.
\begin{table}[!ht]
\centering
\begin{tabular}{|p{2.5in}|p{3.5in}|}
\hline
\textbf{Faults} & \textbf{Possible Soultions} \\ \hline
No sound from Headphone or sound from only one side of the Headphone & Change the Headphone and Check.\\ 
Sound does no go from the Mic of the Headphone &  Clean Headphone Jack and Connector.\\ 
                & Resolder or Change the Headphone Connector.\\ 
                & Check Track of Headphone Section. Do Jumper Wherever required. \\
\hline
\end{tabular}
\caption{Faults \& Solutions of the Headphone.}
\end{table}

\end{itemize}

\section{Computer Hardware, Software and IT Maintenance}
During my training​ I worked on a number of internal components of a computer like ​mother board, power supply unit, RAM, Hard Disk Drive, CDROM, Floppy drive, CPU, Cable connectors, processor fan, external device jacks like VGA, USB, parallel \& serial ports.​ And I was able to repair some computers (System Units and Laptops) which had a problem and servicing and maintenance and software installations.
\subsection{Servicing computers and maintenance}
Computer servicing and maintenance is the physically cleaning of the interior a​nd  ​exterior of a computer, including the removal of dust and debris from cooling fans, power supplies, and other hardware components, performing diagnostic testing, and making repairs, the detection and removal of computer viruses, update software, install firewalls and security programs, upgrade computer memory, or connect and configure Internet and network connections. During servicing and maintenance a number of tools are used and these include: \\ \\
\textbf{Tools used during servicing and maintenance.} \\ 
\begin{itemize}
\item Electrostatic discharge (ESD) protection kit, including wrist strap and mat 
\item Electric screwdriver.
\item Flashlight
\item Wire cutter or stripper
\item Cleaning swabs, canned air (dust blower), and contact cleaner chemicals
\item Data transfer cables and adapters
\item Diagnostics software
\item Markers, pens, and notepads
\item Spare screws, jumpers, standoffs

\end{itemize}
\textbf{Steps taken during Hardware servicing and maintenance} \\
\begin{itemize}
\item[Step 1-]Before opening the outer case, first unplug the PC if it’s in power.
\item[Step 2-]Then Ground yourself before you touch anything inside to avoid destroying your circuitry with a static charge. If you don't have a grounding wrist strap, you can ground yourself by touching any of various household objects. 
\item[Step 3-]Then use a blower to all the dust in the computer. Mainly on components like the fans, heat sinks, in the corners of the casing, video cards, ATX cable adapters, on the board and other parts. 
\item[Step 4-]Use antistatic wipes to remove the remaining dust from inside the case. Avoid touching any circuit­board surfaces.
\item[Step 5-]Check the expansion slots, unplug them clean the with a brush and plug them back well. After servicing the hardware parts there is need to do software services to maximum computer performance so the next thing to do is perform software maintenance. 
\end{itemize}
\textbf{What was done during software maintenance and servicing computer performance optimization }\\
\begin{itemize}
\item Scanning all computer file and removing virus infections. Before you start on any maintenance program, the first thing you need to do is make sure that you have a clean and healthy computer 
\item Test your hard drive for errors.
\item Remove Unwanted Applications, Ad-ons and Toolbars.
\item Prevent Unnecessary Applications and Services from Loading on Start­up. By disabling them if you don’t want to remove them.
\item Clean up Temp Files this helps to delete unnecessary files – This was done using Windows Disk Clean up utility.
\item Performed drive fragmentation using the Windows Defrag utility.
\item Scan File System for Errors​ –​ for this I used the System File Checker which scans the hard drive for corrupt or missing system files and attempts to replace them.
\end{itemize}

\subsubsection{Software installation and updates}
Computer software or simply software is any set of machine-readable instructions that directs a computer's processor to perform specific operations. There mainly two types of softwares and these include:\\ \\
\textbf{Application software}\\
Application software is the one which uses the computer system to perform special functions or provide entertainment functions beyond the basic operation of the computer itself. \\
\textbf{System software}\\
This is software designed to directly operate the computer hardware, to provide basic functionality needed by users and other software, and to provide a platform for running application software. System software includes: \\
\textbf{Operating systems (OS)}\\
 OS are essential collections of software that manage resources and provides common services for other software that runs "on top" of them. The OS is composed of some core parts such as:\\
 \begin{itemize}
 \item Supervisory programs.
 \item Boot loaders. 
 \item Shells. 
 \item Window systems.
 \end{itemize}
In practice, an operating system comes bundled with additional software (including application software) so that a user can potentially do some work with a computer that only has an operating system. E.g. windows 10, 7, 8, Linux, window XP\\
\textbf{Device drivers}\\
These are software’s which operate or control a particular type of device that is attached to a computer. A driver provides a software interface to hardware devices, enabling operating systems and other computer programs to access hardware functions without needing to know precise details of the hardware being used. \\
Drivers interface with a number of hardware devices. \\
\begin{itemize}
\item printers.
\item video adapter.
\item Network cards.
\item Sound cards.
\item Computer storage devices such as hard disk, CD­ROM, and floppy disk buses.
\item Implementing support for different file systems.
\item Image scanners.
\item Digital cameras. 
\end{itemize}
\textbf{Utilities}
These are either applications or computer programs designed to assist users in maintenance and care of their computers. E.g. registry cleaners, disk defragmenters.\\
\textbf{Malicious software}\\
"Malware" is short form for “malicious software” - computer programs designed to infiltrate and damage computers without the users consent. “Malware” is the general term covering all the different types of threats to your computer safety such as viruses, spyware, worms, trojans, rootkits and so on. 

\subsubsection{What is done before installing software?}
The first thing to do before installing any software is to check System Requirements of your computer whether it meets the minimum system requirements for that product/​​software.  And if you are going to install an operating system on a computer being used first backup all the data to avoid any data loss. \\
Install all system updates and reboot your computer. The final step before beginning installation is to exit all applications that are currently running on your system.

\begin{table}
\centering
\begin{tabular}{|l|p{4.5in}|} % 3 cols (left, ctr, right); vert. lines
% draw horizontal line
\hline
Step 1 & Insert the Windows DVD into your computer's DVD­ROM drive, and restart the computer. Windows  Setup should start automatically. If Setup does not start automatically, ensure that your computer is configured to boot from the DVD drive. \\
\hline
Step 2 & In the next dialog box, you are prompted to start the installation. Click Install Now to begin the installation. This produces a screen that tells you that Setup is starting.  \\
\hline
Step 3 & In the Software License Terms dialog box, ensure that you read and understand the End User Licensing Agreement (EULA). Then select the I Accept the License Terms option and click Next to continue. \\
\hline
Step 4  & Select the Type of Installation You Want. You can select only the Custom (Advanced) option because you're performing a new installation on a blank hard disk. Click Custom (Advanced) to continue. \\
\hline
Step 5 & You choose the partition on which You Want to Install Windows, then click Next.Then the Installing Windows dialog box appears and gives you an updated status of the upgrade process. During this process the computer restarts a couple of times, before the restart, a warning appears.  \\
\hline
Step 6 & After completing the installation, Windows 7 asks you to provide a username and a computer name. After providing this information, click Next to continue.\\
\hline
Step 7 & The next step is to activate windows. Get the key on the cover and type it in the space. 
And lastly make the necessary updates, set the time zone and the location then restart the computer. \\

\hline
\end{tabular}
\caption{Steps for the installation of Windows.}
\end{table}

\section{Projects that I worked due to internship connections}
During my internship, my field supervisor connected me with different people so in that process I got freelance projects that I also worked on which included:
\medskip
\begin{itemize}
\item[1.]\textbf{Web deveploment:} There is a NGO called Mash Child Hope For Africa which is just starting and I am developing for them there website which is still under development up to date.
\item[2.] \textbf{SEO:} I did search engine optimisation for a love spell company based in South Africa and also a company based here in Uganda called \textbf{Katrax Bestway Limited}.
\item[3.] \textbf{Google Forms:} An employee of the company that was working as the social media supervisor had to make a daily report on how many people have liked our facebook page with the influence of our sale representatives from the different shops so I advised him to use google forms but he didnot know about them I helped him design the form and made his work more.  
\end{itemize}